%   ------------------------------------------------------------------------
\FloatBarrier
\section{Visão geral da análise comparativa}
\label{s.visaoAnalise}

Como já foi mencionado, a aplicação de IA para a criação de animações 2D não foi muito explorada, tendo um potencial muito grande a ser descoberto. Atualmente, a maioria das ferramentas generativas de vídeo é voltada para ambientes tridimensionais e realistas. Diante desse cenário, esta análise busca investigar a capacidade e o resultado das tecnologias atuais quando aplicadas ao contexto da animação 2D para um jogo.

Durante a análise preliminar, uma das ferramentas foi imediatamente descartada do estudo por não ser capaz de gerar resultados. A ferramenta AI Sprite Sheet Maker, encontrada na plataforma segmind, foi inicialmente selecionada por seu foco na criação do sprite sheet de um personagem a partir de uma única imagem. A funcionalidade apresentada na página da ferramenta (Figura \ref{fig:segmindDemo} do Apêndice \ref{ap.telasIA}) indicava a geração do personagem anexado em diferentes posições, não formando nenhuma ação específica. Esse é um recurso com potencial para a criação de imagens de referência, embora não tenha capacidade de geração direta de animações. A plataforma segmind disponibiliza \$1 de crédito gratuito, enquanto o custo por geração com este modelo é de aproximadamente \$0.01 (Figura \ref{fig:segmindLimitado} do Apêndice \ref{ap.telasIA}). Em teoria, o saldo inicial seria o suficiente para múltiplos testes, porém, ao tentar gerar o sprite sheet, o sistema retornou uma mensagem de erro informando que os créditos eram insuficientes. Diante da impossibilidade de continuar a análise e teste da ferramenta, a mesma foi descartada do estudo. As capturas de tela da interação completa podem ser consultadas na Figura \ref{fig:segmind1} do Apêndice \ref{ap.telasIA}.  


Nas seções seguintes, é apresentada uma análise detalhada das demais ferramentas, sendo o objetivo desse capítulo responder a uma série de questões-chave:

\begin{itemize}
    \item Avaliar se ferramentas com foco em realismo podem ser adaptadas para a animação 2D; 
    \item Analisar o nível de desenvolvimento das ferramentas que possuem foco em 2D; 
    \item Determinar o grau de consistência que as IAs mantêm em relação a um design de personagem pré-existente e a um estilo artístico específico; 
    \item Verificar a possibilidade de utilizar ferramentas de geração de imagem para auxiliar na animação, incluindo a criação sequencial de quadros e a geração de novas poses ou vistas do personagem (como a vista lateral a partir da frontal); e
    \item Investigar a capacidade das ferramentas de gerar uma imagem pixel perfect, característico do estilo pixel art.
\end{itemize}

Ao final, busca-se mostrar o papel prático dessas tecnologias no processo de desenvolvimento de um jogo, posicionando-as não como uma possível substituição ao trabalho artístico, mas como ferramentas potenciais para otimizar e facilitar o complexo processo de animação.

