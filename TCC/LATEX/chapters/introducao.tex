\chapter{Introdução}
\label{c.introducao}


A saúde mental é um componente crítico do bem-estar geral e ela se manifesta de forma variável em cada pessoa, de maneira muito parecida com a saúde física \cite{deleene}. Ao longo da vida, diversos determinantes contribuem para proteger, prejudicar ou deixar mais vulnerável a saúde mental do ser humano, como fatores psicológicos e biológicos individuais, circunstâncias sociais  e ambientes desfavoráveis \cite{world_2022}. 

Esse tema ganhou importância e visibilidade recentemente com a divulgação da Agenda de Desenvolvimento Sustentável 2030 da Organização das Nações Unidas (ONU), pois pela primeira vez as metas incluíram saúde mental de maneira explícita. No Objetivo de Desenvolvimento Sustentável (ODS) 3.4 foi estabelecida a meta de "promover a saúde mental e o bem-estar". Esse tema possui grandes impactos na qualidade de vida humana, pois pesquisas mostraram que condições de saúde mental são responsáveis por 13\% dos anos vividos com incapacidade e perdidos por morte prematura \cite{heymann_sprague_2023}; em 2019, uma a cada oito pessoas viveram com algum transtorno mental, um número que aumentou consideravelmente por causa da pandemia do COVID-19 \cite{mentalDisorders_2022}. 

Jogos Sérios são jogos cujo propósito não é apenas entretenimento, mas também a exploração ativa de problemas sociais \cite{abt1987serious}. Eles são utilizados para diversas finalidades: auxiliar no processo educacional, ajudar pacientes a entender sua condição atual e sua reabilitação, promover a conscientização do público para problemas psicológicos e emocionais, etc. Além disso, uma categoria recente de jogos vem ganhando destaque: jogos empáticos. Esses jogos priorizam, através de mecânicas, trazer a experiência de como é estar no lugar de outra pessoa. O aspecto interativo que os jogos trazem permite que o jogador participe ativamente do conteúdo mostrado, não sendo apenas observadores passivos, mas sim participantes afetados pelos eventos do universo. Dessa forma, é possível fazer com que o usuário tenha interesse em tópicos relacionados à saúde mental (como o luto e transtornos mentais) e ter uma visão mais empática sobre esse tema \cite{10.1145/3638067.3638104}.

O desenvolvimento de jogos 2D envolve a criação de um jogo que exista num espaço bidimensional, onde todos os componentes são representados usando dois eixos. De acordo com \citeonline{article}, apesar de estarmos numa era dominada por gráficos 3D, os jogos 2D mantêm sua popularidade. Isso ocorre pois são mais baratos de serem produzidos, sendo o melhor mercado para um desenvolvedor independente \cite{book}.

A Inteligência Artificial (IA) é um campo rico e diverso, que possui aplicabilidade em várias vertentes como automóveis, saúde, entretenimento, educação, segurança, entre outras. A IA foca em aprender com experiências e alterar seu processamento e comportamento baseado em seu aprendizado. Além disso, é considerada a próxima revolução industrial na área de entretenimento e também é capaz de aumentar a eficiência automatizando numerosas tarefas repetitivas \cite{articleIAEntretenimento}. Na indústria de jogos, a IA é utilizada para  gerar comportamentos responsivos, adaptativos e inteligentes para personagens não jogáveis (em inglês, NPCs) \cite{IA_jogos}. De acordo com \citeonline{jorapur2024evolution}, ela também é implementada para adaptar a história dependendo do comportamento do jogador, para criação de diferentes conteúdos do jogo como níveis e mundos infinitos e para ajustar a dificuldade de acordo com a performance do player. Além disso, é usada para fazer animações de personagem \cite{xian2023automated}.

Nesse contexto, o projeto visa desenvolver um jogo sério 2D que utiliza IA para fazer as animações dos personagens, e que aborda temas de saúde mental para trazer uma visão mais empática e conscientizar sobre esse assunto. 







 
 
  