\chapter{Problema}
\label{c.problema}

De acordo com a Organização Mundial da Saúde (OMS), a saúde mental é extremamente importante para todos. Globalmente, as necessidades de saúde mental são altas, mas as respostas são insuficientes e inadequadas \cite{world}.

Transtornos mentais, em muitos casos, não são facilmente reconhecíveis baseados na aparência e comportamento externo de uma pessoa. Assim, aqueles afetados por isso sofrem com a falta de empatia e de aceitação, sendo alvos de discriminação e psicofobia (preconceito direcionado a pessoas com transtornos mentais) \cite{kasdorf_2023}. O estigma é encontrado no ambiente de trabalho, familiar e escolar, fazendo com que o indivíduo se retraia, seja desencorajado a discutir sobre sua saúde mental, desenvolva mecanismos de enfrentamento prejudiciais e tenha um agravamento dos sintomas. Além disso, existe também a  auto estigmatização, que é a internalização de crenças negativas sobre si mesmo e auto discriminação \cite{Roma_2024}.

A IA tem feito contribuições significativas para os jogos digitais, porém sua aplicação em animações 2D é pouco explorada. As ferramentas de IA para animação atualmente têm como alvo especificamente vídeos tridimensionais, que não têm um controle efetivo sobre a aleatoriedade \cite{articleIAanima}. 







