\chapter{Justificativa}
\label{c.justificativa}

 A mídia em geral é uma fonte extremamente importante de conhecimento sobre transtornos, com os jogos sendo um dos tipos mais impactantes de mídia digital. Essa representação de assuntos relacionados à saúde mental afeta as atitudes e crenças do público. No mundo dos jogos digitais, por exemplo, é muito comum a representação de transtornos mentais, porém ela é acompanhada de discriminação e estigmatização \cite{kasdorf_2023}. 

 Conforme citado, o uso da IA nas animações 2D foi muito pouco explorado, fazendo com que possivelmente essa tecnologia tenha o potencial de revolucionar o processo complexo de produção de animação 2D \cite{articleIAanima}. Automatizando o processo, o tempo e o esforço são reduzidos, sendo possível produzir animações de alta qualidade mais rapidamente e com um custo menor \cite{xian2023automated}.

Diante do contexto apresentado, é necessário encontrar maneiras de combater os estigmas e preconceitos envolvidos ao redor do tema de saúde mental, além de ser necessário disseminar mais sobre o assunto e sua importância. Para isso, a proposta desse projeto é desenvolver um jogo sério 2D que aborda temas de saúde mental de forma mais subjetiva, cujo jogador controlará um personagem que passa por problemas psicológicos e emocionais. Além disso, será usada uma IA para fazer as animações do jogo e analisar o potencial dessa tecnologia para o auxílio na produção de animações 2D. O projeto também proporciona uma oportunidade para praticar os conhecimentos de Computação Gráfica e Engenharia de Software, contribuindo para a formação acadêmica.

